\documentclass{standalone}
\usepackage{tikz}
\usepackage{pgfplots}

\pgfplotsset{compat = 1.15}


\pgfmathdeclarefunction{gauss}{3}{%
  \pgfmathparse{1/(#3*sqrt(2*pi))*exp(-((#1-#2)^2)/(2*#3^2))}%
}

\begin{document}

\begin{tikzpicture}
    
    \draw[white] (-10, -10) -- (10, -10) -- (10, 10);

    %\draw[step = 1.0cm, black, ultra thin] (0,0) grid(20, 20);
    
    \begin{scope}[xshift = 0]
    
    % Circle one
    
    \draw[thick] (0,0) circle (3cm);
    \draw[fill=blue!40!white, draw=black] (-1,0) circle (0.5cm);
    \draw[fill=red!40!white, draw=black] (1,0) circle (0.5cm);
    
    \end{scope}
    
    %\begin{scope}[xshift = 25cm, yshift = -20.0cm,  rotate around= {-45:(24,14)}]
    \begin{scope}[xshift = 0cm, yshift = 0cm,  rotate around= {225:(0,0)}]
    
    % Camera-config 1
    
    \draw[thick, fill=gray] (-2, 8) rectangle (2,9);
    
    % Camera-beam 1
    
    \draw[ultra thick] (0, 8) -- (-3, -4);
    \draw[ultra thick] (0, 8) -- (3, -4);
    \draw[ultra thick] (-3, -4) -- (3, -4);

    % Gaussian one
    
    \begin{axis}[rotate around={180:(current axis.origin)},
        anchor = origin,
        axis lines*=center,
        x = 1cm,
        at = {(7.0cm, -1.7cm)},
        xtick=\empty, ytick=\empty,
        style = {samples = 300, smooth},
        height = 3cm,
        width = 2cm,% scale only axis,
        hide axis
        ]
        
        \addplot [mark = none, blue, restrict x to domain = -2:2] {gauss(x, 0, 0.25)};
        
        %\addplot [mark = none, blue, restrict x to domain = 2:3] %{gauss(x, 1, 1)};
    
    \end{axis}
    
      \begin{axis}[rotate around={180:(current axis.origin)},
        anchor = origin,
        axis lines*=center,
        x = 1cm,
        at = {(4.5cm, -1.7cm)},
        xtick=\empty, ytick=\empty,
        style = {samples = 300, smooth},
        height = 3cm,
        width = 2cm,% scale only axis,
        hide axis
        ]
        
        \addplot [mark = none, red, restrict x to domain = -2:2] {gauss(x, 0, 0.25)};

    
    \end{axis}
    
    \end{scope}
    

\end{tikzpicture}

\end{document}
