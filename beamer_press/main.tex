\documentclass{beamer}[10pt]
%\usetheme{metropolis}           % Use metropolis theme - has to compile with the xeTex (can be done on sharelatex) 

\usetheme[progressbar=frametitle]{metropolis} % Have a progressbar instead
\usepackage{appendixnumberbeamer}

%\usepackage[utf8]{inputenc}

\newenvironment{wideitemize}{\itemize\addtolength{\itemsep}{10pt}}{\enditemize}

\title{Kvantitativ SPECT - Dosimetri}
\date{\today}
\author{Johan Blakkisrud}
\institute{Avd. for Diagnostisk Fysikk, Oslo Universitetssykehus}
\begin{document}

\maketitle

\section{Introduksjon}

\begin{frame}{Hvem er jeg?}

\begin{itemize}
    \item 
    Stipendiat ved Oslo Universitetssykehus (p� ca tredje �ret)
    \item
    Fysiker (egentlig siv. ing. fra NTNU)
    \item
    Prosjektet jeg jobber i bruker SPECT/CT-bilder av den kvantitative formen
\end{itemize}

\end{frame}

\begin{frame}{Kvantitering}

    Kvantitering (verb) - m�le, telle. 
    Tilordne til en \emph{kvantitet}

\end{frame}

\begin{frame}{Kvantitering}

    \begin{columns}

	\begin{column}{0.5 \textwidth}

	    \center
	    \begin{itemize}
		\item<1->
		    Infinitiv (hint: sett � foran)
		\item<3->
		    Presens (hint: hva vi gj�r n�)
		\item<5->
		    Preteritum futurum perfektum, Kondisjonalis II (hint: hva vi ikke vil si)
	    \end{itemize}

	\end{column}

	\begin{column}{0.5 \textwidth}

	    \center
	    \begin{itemize}
		\item<2->
		    Kvantitere 
		\item<4->
		    Kvantiterer
		\item <6->
		    Ville/skulle ha kvantitert
	    \end{itemize}
	    
	\end{column}
    \end{columns}

\end{frame}

\begin{frame}{Kvantitere ifm. SPECT}

    \center
    Avbildning der bildene har enhet MBq/ml eller lignende

\end{frame}

\begin{frame}{Et dogme}

    \center
    PET er kvantitativt - SPECT er det ikke!

\end{frame}

\begin{frame}{Kvantitering mindre viktig}

\end{frame}

\begin{frame}{Kvantitering viktig}

        %Kvantitering viktig\\[.2cm]
    \begin{description}
        \item
            [Terapiplanlegning] Dosering av radionuklideterapi
        \item
            [Myocard-perfusjonsavbildning] 99m-Tc for � m�le blodgjennomstr�mning i ml/g-min eller SUV-SPECT
        \item
            [Lungescan] Pre-operativ funksjon b�de ventilasjon og perfusjon
        \item
            [131-I] Opptak post-terapi for � monitorere effekt
        \item
            [Biodistribusjon] Utpr�ving av nye radiofarmaka
    \end{description}
    
\end{frame}

\section{SPECT-kameraet}

\begin{frame}{Et lite tilbakeblikk}

    \begin{center}
    \only<1>{\resizebox{\textwidth}{!}{
	    \includegraphics{graphics/timeline/time0}
    }}
    \only<2>{\resizebox{\textwidth}{!}{
	    \includegraphics{graphics/timeline/time1}
    }}
    \only<3>{\resizebox{\textwidth}{!}{
	    \includegraphics{graphics/timeline/time2}
    }}
    \only<4>{\resizebox{\textwidth}{!}{
	    \includegraphics{graphics/timeline/time3}
    }}

    \end{center}

\end{frame}

\begin{frame}{SPECT-kameraet}
\begin{columns}[T]
\begin{column}{.58\textwidth}
    \only<1>{\resizebox{\textwidth}{!}{
	    \includegraphics{graphics/camera_rot/img0}
    }
  }
    \only<2>{\resizebox{\textwidth}{!}{
      \includegraphics{graphics/camera_rot/img1}
    }
  }
    \only<3>{\resizebox{\textwidth}{!}{
      \includegraphics{graphics/camera_rot/img2}
    }
  }
    \only<4>{\resizebox{\textwidth}{!}{
      \includegraphics{graphics/camera_rot/img3}
    }
  }
    \only<5>{\resizebox{\textwidth}{!}{
      \includegraphics{graphics/camera_rot/img4}
    }
  }
    \only<6>{\resizebox{\textwidth}{!}{
      \includegraphics{graphics/camera_rot/img5}
    }
  }
\end{column}
\hfill
\begin{column}{.4\textwidth}
  \begin{wideitemize}
  \item<1-> To radiaktive kilder
  \item<2-> SPECT-kameraet tar bilde av en projeksjon
  \item<3-> Roterer og tar et nytt bilde
  %\item<4-> 
  %\item<5-> 
  %\item<6-> 
  \end{wideitemize}
  \end{column}
\end{columns}
\end{frame}

\begin{frame}{Attenuasjon og spredning}

    \begin{center}
	\only<1>{\resizebox{0.88\linewidth}{!}{
      \includegraphics{graphics/attenuation/att0}
    }}
    \only<2>{\resizebox{0.88\linewidth}{!}{
      \includegraphics{graphics/attenuation/att1}
    }}
    \only<3>{\resizebox{0.88\linewidth}{!}{
      \includegraphics{graphics/attenuation/att2}
    }}
    \end{center}

\end{frame}
    

\begin{frame}{Partiell volumeffekt}
    
    \begin{center}
	\resizebox{0.88\linewidth}{!}{
      \includegraphics{graphics/lit_figs/PartialVolume}
    }

    \end{center}
    
\end{frame}

\begin{frame}{PET kvantitativ, men ikke SPECT?}

\begin{itemize}
    \item 
        Spredt str�ling (var) et st�rre problem (<5 \% i tidlig PET, n� 35 - 50 > \% for b�de 3D PET og SPECT)
    \item
        PET ble "gjemt bort" i 20 �r som forskningsobjekt - kinetikk med arteriell aktivitetskonsentrasjon som input
    \item
        Attenueringskorreksjon (var) mer rett frem i PET
        
\end{itemize}
    
\end{frame}

\section{Fantomstudier - hvor er SPECT n�?}

\begin{frame}{Fantomstudier}

    \begin{center}
	\only<1>{\resizebox{0.58\linewidth}{!}{
      \includegraphics{graphics/lit_figs/phantom_ilus_1}
    }}
	\only<2>{\resizebox{0.58\linewidth}{!}{
      \includegraphics{graphics/lit_figs/phantom_ilus_2}
    }}
	\only<3>{\resizebox{0.58\linewidth}{!}{
      \includegraphics{graphics/lit_figs/phantom_ilus_3}
    }}
    \end{center}
    
\end{frame}


\begin{frame}{Fantomstudier}

    \begin{center}
    \begin{alertblock}{}
        Dette er et veldig vanskelig sp�rsm�l!
    \end{alertblock}
    \end{center}
    
\end{frame}

\begin{frame}{Fantomstudier}

\begin{quote}
    99mTc, 111In, 131I "To summarize: activity can be accurately meassured \textbf{(+/- 10 \%)} at least in uniformely attenuating parts of the body,..., as small as 40 ml"
\end{quote}

- Quantitative SPECT in radiation dosimetry (1989)
    
\end{frame}

\begin{frame}{Fantomstudier}

\begin{quote}
    "...the results should be improved image quality and, perhaps, quantitative \textbf{accuracies of about 10 \%}"
\end{quote}

- Quantitative SPECT Imaging (1995)
    
\end{frame}

\begin{frame}{Fantomstudier}

    \begin{quote}
        "quantification based on a calibration procedure similar to the one used in PET seems to be feasible \textbf{within 10 \% error limits and even below } if a fine-tuning of all acquisition and reconstruction parameters is performed."
    \end{quote}
    
    - Quantitative capabilities of four state of the art SPECT/CT-systems (2012)
    
\end{frame}

%\begin{frame}{Enkeltstudier}

%\begin{columns}

%\begin{column}{0.5 \linewidth}

%\begin{description}
%    \item
%        [99mTc] - Kort beskrivelse
%    \item
%        [131-I] - Kort beskrivelse
%    \item
%        [177-Lu] - Kort beskrivelse
%    
%\end{description}
%
%\end{column}}

%\end{columns}}
    
%\end{frame}

\begin{frame}{Fantomstudier}

Det kommer an p� en hel rekke ting:

\begin{itemize}
    \item 
        Nukliden
    \item
        Organet man vil m�le i (st�rrelse, form og plassering)
    \item
        Bildeprotokoll
    \item
        Kalibreringsmetoden
    \item
        Segmentering
    \item
        ...
        
\end{itemize}

\end{frame}

\begin{frame}{99mTc - m/ xSPECT }

    \begin{center}
	\only<1>{\resizebox{0.88\linewidth}{!}{
      \includegraphics{graphics/lit_figs/ArmstrongFig2}
    }}
    
	\only<2>{\resizebox{0.88\linewidth}{!}{
      \includegraphics{graphics/lit_figs/ArmstrongFig3}
    }}
%
    \end{center}
%
\end{frame}

\begin{frame}

\end{frame}

\begin{frame}{Fantomstudier}

Hva kan du forvente?

\begin{description}
    \item
    [Tommelfingerregel] Omkring 10 \%
    \item
    [Om du vet hva du gj�r] Under omkring 5 \%
    \item
    [Om du \textbf{ikke} vet hva du gj�r] Mer enn 50 \%
\end{description}
    
\end{frame}

\section{SUV-SPECT}

\begin{frame}{Benvekst}

    \begin{center}
	\only<1>{\resizebox{0.88\linewidth}{!}{
      \includegraphics{graphics/lit_figs/YamaneFig1}
    }}
    \end{center}
	
\end{frame}

\begin{frame}{Benvekst}

    \begin{center}
	\only<1>{\resizebox{0.88\linewidth}{!}{
      \includegraphics{graphics/lit_figs/YamaneFig2}
    }}
    \end{center}
	
\end{frame}

\begin{frame}{Benvekst}

    \begin{center}
	\only<1>{\resizebox{0.88\linewidth}{!}{
      \includegraphics{graphics/lit_figs/YamaneFig3}
    }}
    \end{center}
	
\end{frame}

\begin{frame}{Thyrodea}

    \begin{center}
	\only<1>{\resizebox{0.88\linewidth}{!}{
      \includegraphics{graphics/lit_figs/KimFig1}
    }}
    \end{center}
	
\end{frame}

\begin{frame}{Thyrodea}

    \begin{center}
	\only<1>{\resizebox{0.88\linewidth}{!}{
      \includegraphics{graphics/lit_figs/KimFig2}
    }}
    \end{center}
	
\end{frame}
% A novel SUV-based quantification of 99mTc-MDP SPECT/CT uptake and 18F-FDG % PET/CT metabolism in patients with lumbar disc herniation

% SUV measurement of normal vertebrae using SPECT/CT with Tc-99m methylene % diphosphonate


\section{xSPECT}
\section{Dosimetri}
\section{Eget arbeid}

\begin{frame}{Comments}

\begin{quote}
    Gir bildene en ekstra nytteverdi - selv om man ikke bruker dem hver gang er de fine � ha. 
\end{quote}
    
\end{frame}

\begin{frame}{Comments}

\begin{quote}
    Life without you would be like a broken pencil... pointless! 
\end{quote}
- Edmund Blackadder
    
\end{frame}




\end{document}
